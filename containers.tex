\documentclass{dcpresentation}

% Presentation info
\title{Containers for Dummies}
\author{}
\newcommand{\authorEmail}{}
\institute{SciLifeLab Data Centre}
\date{}

\usepackage{listings}

\usepackage{xcolor}

\definecolor{commentsColor}{rgb}{0.497495, 0.497587, 0.497464}
\definecolor{keywordsColor}{rgb}{0.000000, 0.000000, 0.635294}
\definecolor{stringColor}{rgb}{0.558215, 0.000000, 0.135316}
 
\lstset{ %
  backgroundcolor=\color{white},   % choose the background color; you must add \usepackage{color} or \usepackage{xcolor}
  basicstyle=\footnotesize,        % the size of the fonts that are used for the code
  breakatwhitespace=false,         % sets if automatic breaks should only happen at whitespace
  breaklines=true,                 % sets automatic line breaking
  captionpos=b,                    % sets the caption-position to bottom
  commentstyle=\color{commentsColor}\textit,    % comment style
  deletekeywords={...},            % if you want to delete keywords from the given language
  escapeinside={\%*}{*)},          % if you want to add LaTeX within your code
  extendedchars=true,              % lets you use non-ASCII characters; for 8-bits encodings only, does not work with UTF-8
  keepspaces=true,                 % keeps spaces in text, useful for keeping indentation of code (possibly needs columns=flexible)
  keywordstyle=\color{keywordsColor}\bfseries,       % keyword style
  language=Python,                 % the language of the code (can be overrided per snippet)
  otherkeywords={*,...},           % if you want to add more keywords to the set
  rulecolor=\color{black},         % if not set, the frame-color may be changed on line-breaks within not-black text (e.g. comments (green here))
  showspaces=false,                % show spaces everywhere adding particular underscores; it overrides 'showstringspaces'
  showstringspaces=false,          % underline spaces within strings only
  showtabs=false,                  % show tabs within strings adding particular underscores
  stepnumber=1,                    % the step between two line-numbers. If it's 1, each line will be numbered
  stringstyle=\color{stringColor}, % string literal style
  tabsize=2,                      % sets default tabsize to 2 spaces
  title=\lstname,                  % show the filename of files included with \lstinputlisting; also try caption instead of title
  columns=fixed                    % Using fixed column width (for e.g. nice alignment)
}



\begin{document}

  % put logo in upper right corner to ape the official template
 \AddToShipoutPictureFG{
  \AtPageUpperRight{{\includegraphics[width=0.6cm,keepaspectratio]{img/scilifelab-symbol.pdf}}}
 }%

 \section{Introduction}

 \begin{frame}{Goals for Today}
  \begin{itemize}
   \item General orientation about how we (developers/sysadmin) use containers for development and deployment
   \item Differences between emulators, virtual machines, and containers
   \item Guided tour of DC-kube, our Kubernetes cluster
  \end{itemize}
 \end{frame}

 \begin{frame}{The Dependency Issue}
  \begin{itemize}
   \item Software written for one computer architecture can't be run on another
   \item All software require a specific set of libraries to run
   \begin{itemize}
    \item Common that different software requires the same library but different versions
   \end{itemize}
  \end{itemize}
 \end{frame}

 \begin{frame}{Add operation opcode}{Performing addition using different architectures}
  
  \begin{description}
   \item[ARM] 0100
   \item[MIPS] 100011
   \item[x86] 000000xy
  \end{description}
 \end{frame}

 \section{Emulators}

 \begin{frame}{Emulators}
  \begin{itemize}
   \item Translate machine code
   \item Run software for x86 on ARM
   \item "Slow"
   \item QEMU
  \end{itemize}
 \end{frame}
 
  \section{Virtual Machines} 
 
 \begin{frame}{Virtual Machines}
  \begin{itemize}
   \item Abstract hardware
   \item Can be transferred between physical computers
   \item Requires dedicated hardware
   \item "Long-term"
   \item QEMU
   \item KVM
   \item VirtualBox (x86 only)
   \item May run at almost the same speed as the host
  \end{itemize}
 \end{frame}
  
 \begin{frame}
  \centering \alert{Demo: QEMU and virtual machines} 
 \end{frame}   
 
 \section{Containers} 
  
 \begin{frame}{Containers}
  \begin{itemize}
   \item Namespace isolation
   \item Sharing hardware with all other containers and the host
   \item Images --- "templates"
   \item "Short-term"
   \item No host --- runs normally on the computer
  \end{itemize}
 \end{frame}
 
 \begin{frame}{Containers}{Software}
  \begin{columns}
   \column{0.6\textwidth}
    \begin{itemize}
     \item Open Container Initiative (OCI)
     \item Repositories
     \begin{itemize}
      \item Dockerhub
      \item Github Container Repository   
     \end{itemize}
    \end{itemize}
   \column{0.4\textwidth}
    \begin{itemize}
     \item Docker
     \item Podman
     \item Containerd
     \item Singularity
     \item LXC
    \end{itemize}
  \end{columns}
 \end{frame}

 \subsection{Docker}

 \begin{frame}
  \begin{center}
   \alert{\Large Demo}
   \vspace{7pt}

   \alert{Docker Desktop/Rancher Desktop}
   \vspace{3pt}
   
   \alert{Using a container}
   \vspace{3pt}
   
   \alert{Dockerfile}
   \vspace{3pt}
   
   \alert{Docker-compose}
  \end{center}
 \end{frame}

 \section{Kubernetes}

 \begin{frame}{Kubernetes}
  \begin{itemize}
   \item Container orchestration
   \item Originates from Google
   \item "docker-compose for clusters"
  \end{itemize}
 \end{frame}  

 \begin{frame}
  \centering \alert{Demo: DC-Kube}
 \end{frame}

\end{document}
